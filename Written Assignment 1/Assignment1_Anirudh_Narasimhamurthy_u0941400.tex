%-*- Mode:LaTeX; -*-      
\documentclass[11pt]{article}
\usepackage{vmargin}		% Force narrower margins
\setpapersize{USletter}
\setmarginsrb{1.0in}{1.0in}{1.0in}{0.6in}{0pt}{0pt}{0pt}{0.4in}
\setlength{\parskip}{.1in}  % removed space between paragraphs
\setlength{\parindent}{0in}

\usepackage{epsfig}
\usepackage{graphicx}

\begin{document}

\large
\begin{center}
{\bf CS-5340/6340, Written Assignment \#1} \\
{\bf Anirudh Narasimhamurthy(u0941400)}
\end{center}
\normalsize

\begin{enumerate}  

\item (40 pts, 1/2 pt per word) For each sentence below, label each word with its
  correct part-of-speech (POS) tag based upon the word's use in the sentence.
  Punctuation should be ignored.


\begin{enumerate}

\item Dan slid sideways down the hill and broke out in laughter. \\
\textbf{Solution:} Dan/{\sc noun} slid/{\sc verb} sideways/{\sc adv} down/{\sc prep} the/{\sc art} hill/{\sc noun} and/{\sc conj} broke/{\sc verb} out/{\sc part} in/{\sc prep} laughter/{\sc noun} \\

\item Children may like candy but eating sugary foods is unhealthy. \\
\textbf{Solution:} Children/{\sc noun} may/{\sc mod} like/{\sc verb} candy/{\sc noun} but/{\sc conj} eating/{\sc verb} sugary/{\sc adj} foods/{\sc noun} is/{\sc verb} unhealthy/{\sc adj} \\


\item My brother did not plan to take sleeping pills although he got
  no sleep yesterday.  \\
  \textbf{Solution:} My/{\sc posspro} brother/{\sc noun} did/{\sc verb} not/{\sc adv} plan/{\sc verb} to/{\sc inf} take/{\sc verb} sleeping/{\sc ger} pills/{\sc noun} although/{\sc conj} he/{\sc perpro} got/{\sc verb} no/{\sc adj} sleep/{\sc noun} yesterday/{\sc adv}\\
  

\item He often snores like an aardvark.  \\
\textbf{Solution:} He/{\sc perpro} often/{\sc adv} snores/{\sc verb} like/{\sc prep} an/{\sc art} aardvark/{\sc noun}  \\


\item Mary ran inside after rain began to fall.  \\
\textbf{Solution:} Mary/{\sc noun} ran/{\sc verb} inside/{\sc prep} after/{\sc conj} rain/{\sc noun} began/{\sc verb} to/{\sc inf} fall/{\sc verb}\\


\item The armed man took off when police showed up.  \\
\textbf{Solution:} The/{\sc art} armed/{\sc adj} man/{\sc noun} took/{\sc verb} off/{\sc part} when/{\sc conj} police/{\sc noun} showed/{\sc verb} up/{\sc part}  \\


\item The kittens sleeping in her lap are very young.   \\
\textbf{Solution:} The/{\sc art} kittens/{\sc noun} sleeping/{\sc verb} in/{\sc prep} her/{\sc posspro} lap/{\sc noun} are/{\sc verb} very/{\sc adv} young/{\sc adj}  \\

\item She just completed a singing competition, which could make her a star.  \\
\textbf{Solution:} She/{\sc perpro} just/{\sc adv} completed/{\sc verb} a/{\sc noun} singing/{\sc ger} competition/{\sc noun}, which/{\sc relpro} could/{\sc mod} make/{\sc verb} her/{\sc posspro} a/{\sc art} star/{\sc adj} \\


\end{enumerate}



\newpage
\item (20 pts) For each sentence below, indicate whether the main verb
  appears in an {\it intransitive} construction, a {\it transitive}
  construction, or a {\it ditransitive} construction. Only give the
  answer {\it transitive} if the   usage is \underline{not} {\it ditransitive}.

\begin{enumerate}

\item The dog barked at the cat. \\
\textbf{Solution:} transitive\\

\item The man fed the squirrels peanuts.  \\
\textbf{Solution:} ditransitive\\

\item Susan slept for ten hours. \\
\textbf{Solution:} intransitive\\

\item George broke the window with his fist. \\
\textbf{Solution:} transitive\\

\item Mary loaned her neighbor a bicycle for a week.  \\
\textbf{Solution:} ditransitive\\

\item Ted donated five hundred dollars to his favorite charity. \\
\textbf{Solution:} ditransitive\\

\item Wilma married Fred in a rock quarry.  \\
\textbf{Solution:} transitive\\

\item Sam bought flowers for his mom. \\
\textbf{Solution:} ditransitive\\

\item The cat frequently sits on the front porch. \\
\textbf{Solution:} transitive\\

\item She gave a raise to her best employee for his great work.  \\
\textbf{Solution:} ditransitive\\

\end{enumerate}


\newpage
\item (20 pts) For each sentence below, indicate whether the main verb
  appears in an {\it active} voice verb phrase or a {\it passive}
  voice verb phrase. 

\begin{enumerate}

\item Dr. Seuss has written many books. \\
\textbf{Solution:} Active voice\\

\item Tim will be organizing a charity event. \\
\textbf{Solution:} Active voice\\

\item Cathy has been hired by IBM. \\
\textbf{Solution:} Passive voice\\

\item Walter will be evaluated for a raise in October.  \\
\textbf{Solution:} Passive voice\\

\item Tropical storm Fred has strengthened into a hurricane. \\
\textbf{Solution:} Active voice\\

\item The battle is being fought on several fronts.  \\
\textbf{Solution:} Passive voice\\

\item The cougar was hiding in a bush. \\
\textbf{Solution:} Active voice\\

\item The dog has been taught twenty difficult tricks.  \\
\textbf{Solution:} Passive voice\\

\item She will have achieved a record in gymnastics. \\
\textbf{Solution:} Active voice\\

\item He should have won the award.  \\
\textbf{Solution:} Active voice\\

\end{enumerate}


\newpage
\item (20 pts) Consider the following morphology rules and dictionary:

\begin{center}
\begin{tabular}{|l|l|l|l|l|l|l|} \hline
~ & \textbf{Suffix} & \textbf{Prefix} & \textbf{Replacement} & \textbf{POS of} & \textbf{POS of } \\
~ & ~ & ~ & \textbf{Chars} & \textbf{root word} & \textbf{derived word} \\ \hline
{\it Rule \#1} & s & - & - & NOUN & NOUN \\
{\it Rule \#2} &s & - & - & VERB & VERB \\
{\it Rule \#3} &er & - & - & VERB & NOUN \\
{\it Rule \#4} &- & re & - & VERB & VERB \\
{\it Rule \#5} &- & anti & - & NOUN & ADJECTIVE \\ \hline
\end{tabular}
\end{center}

\begin{center}
\begin{tabular}{|l|l|} \hline
\multicolumn{2}{|c|}{\bf Dictionary} \\ \hline
{\bf Word} & {\bf Part-of-Speech} \\ \hline
seizure & NOUN \\
form & VERB \\ \hline
\end{tabular}
\end{center}

For each word below, indicate whether that word {\it CAN} or {\it CANNOT} be
successfully derived as having the specified part-of-speech using the
morphology rules and dictionary above:

\begin{enumerate}
\item antiseizure ADJECTIVE \\
\textbf{Solution:} CAN\\

\item seizures VERB \\
\textbf{Solution:} CANNOT\\

\item antiseizures NOUN \\
\textbf{Solution:} CANNOT\\

\item antiseizures ADJECTIVE \\
\textbf{Solution:} CAN\\

\item reforms NOUN \\
\textbf{Solution:} CANNOT\\

\item reforms VERB \\
\textbf{Solution:} CAN\\

\item antireform ADJECTIVE \\
\textbf{Solution:} CANNOT\\

\item rereform VERB \\
\textbf{Solution:} CANNOT\\

\item reformer NOUN \\
\textbf{Solution:} CAN\\

\item reformers VERB \\
\textbf{Solution:} CAN\\
 
\end{enumerate}


\underline{\textbf{Question \#5 is for CS-6340 students ONLY!}}  \\

\item (15 pts) Consider the following five subcategorization frames:

\begin{center}
\begin{tabular}{|l|} \hline
NP  \\
NP NP \\
PP(against) \\
PP(from) PP(to) \\
VP(to) \\ \hline
\end{tabular}
\end{center}

For each verb below, list \underline{ALL} of the subcategorization
frames in the list above that should be associated with the verb. 
If a verb should not have ANY of these
subcategorization frames, then give the answer {\it NONE}.

\noindent
HINT: most of the verbs should have 1 or 2 of the subcategorization
frames in the list above.  \\

\begin{enumerate}

\item snore \\
\textbf{Solution:} None\\

\item drive  \\
\textbf{Solution:} NP, VP(to)\\

\item expect \\
\textbf{Solution:} NP, VP(to)\\

\item fight \\
\textbf{Solution:} PP(against), VP(to)\\

\item sip \\
\textbf{Solution:} NP\\

\item sing \\
\textbf{Solution:}  NP, VP(to)\\

\item lean \\
\textbf{Solution:} PP(against), VP(to)\\

\item smile \\
\textbf{Solution:} NP\\

\item lend \\
\textbf{Solution:} NP, NP NP\\
Ex: He lent a book \\
He lent Mary a book(NP NP)

\item increase \\
\textbf{Solution:} PP(from) PP(to), VP(to) \\

\end{enumerate}

\end{enumerate}  % END OF WRITTEN QUESTIONS

\end{document}

